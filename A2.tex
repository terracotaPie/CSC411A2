%%%%%%%%%%%%%%%%%%%%%%%%%%%%%%%%%%%%%%%%%
% Programming/Coding Assignment
% LaTeX Template
%
% This template has been downloaded from:
% http://www.latextemplates.com
%
% Original author:
% Ted Pavlic (http://www.tedpavlic.com)
%
% Note:
% The \lipsum[#] commands throughout this template generate dummy text
% to fill the template out. These commands should all be removed when 
% writing assignment content.
%
% This template uses a Perl script as an example snippet of code, most other
% languages are also usable. Configure them in the "CODE INCLUSION 
% CONFIGURATION" section.
%
%%%%%%%%%%%%%%%%%%%%%%%%%%%%%%%%%%%%%%%%%

%----------------------------------------------------------------------------------------
%	PACKAGES AND OTHER DOCUMENT CONFIGURATIONS
%----------------------------------------------------------------------------------------

\documentclass{article}

\usepackage{fancyhdr} % Required for custom headers
\usepackage{lastpage} % Required to determine the last page for the footer
\usepackage{extramarks} % Required for headers and footers
\usepackage[usenames,dvipsnames]{color} % Required for custom colors
\usepackage{graphicx} % Required to insert images
\usepackage{subcaption}
\usepackage{listings} % Required for insertion of code
\usepackage{courier} % Required for the courier font
\usepackage{lipsum} % Used for inserting dummy 'Lorem ipsum' text into the template

% Margins
\topmargin=-0.45in
\evensidemargin=0in
\oddsidemargin=0in
\textwidth=6.5in
\textheight=9.0in
\headsep=0.25in

\linespread{1.1} % Line spacing

% Set up the header and footer
\pagestyle{fancy}
\lhead{\hmwkAuthorName} % Top left header
\chead{\hmwkClass\ (\hmwkClassTime): \hmwkTitle} % Top center head
%\rhead{\firstxmark} % Top right header
\lfoot{\lastxmark} % Bottom left footer
\cfoot{} % Bottom center footer
\rfoot{Page\ \thepage\ of\ \protect\pageref{LastPage}} % Bottom right footer
\renewcommand\headrulewidth{0.4pt} % Size of the header rule
\renewcommand\footrulewidth{0.4pt} % Size of the footer rule

\setlength\parindent{0pt} % Removes all indentation from paragraphs

%----------------------------------------------------------------------------------------
%	CODE INCLUSION CONFIGURATION
%----------------------------------------------------------------------------------------

\definecolor{MyDarkGreen}{rgb}{0.0,0.4,0.0} % This is the color used for comments
\lstloadlanguages{Perl} % Load Perl syntax for listings, for a list of other languages supported see: ftp://ftp.tex.ac.uk/tex-archive/macros/latex/contrib/listings/listings.pdf
\lstset{language=Perl, % Use Perl in this example
        frame=single, % Single frame around code
        basicstyle=\small\ttfamily, % Use small true type font
        keywordstyle=[1]\color{Blue}\bf, % Perl functions bold and blue
        keywordstyle=[2]\color{Purple}, % Perl function arguments purple
        keywordstyle=[3]\color{Blue}\underbar, % Custom functions underlined and blue
        identifierstyle=, % Nothing special about identifiers                                         
        commentstyle=\usefont{T1}{pcr}{m}{sl}\color{MyDarkGreen}\small, % Comments small dark green courier font
        stringstyle=\color{Purple}, % Strings are purple
        showstringspaces=false, % Don't put marks in string spaces
        tabsize=5, % 5 spaces per tab
        %
        % Put standard Perl functions not included in the default language here
        morekeywords={rand},
        %
        % Put Perl function parameters here
        morekeywords=[2]{on, off, interp},
        %
        % Put user defined functions here
        morekeywords=[3]{test},
       	%
        morecomment=[l][\color{Blue}]{...}, % Line continuation (...) like blue comment
        numbers=left, % Line numbers on left
        firstnumber=1, % Line numbers start with line 1
        numberstyle=\tiny\color{Blue}, % Line numbers are blue and small
        stepnumber=5 % Line numbers go in steps of 5
}

% Creates a new command to include a perl script, the first parameter is the filename of the script (without .pl), the second parameter is the caption
\newcommand{\perlscript}[2]{
\begin{itemize}
\item[]\lstinputlisting[caption=#2,label=#1]{#1.pl}
\end{itemize}
}

%----------------------------------------------------------------------------------------
%	DOCUMENT STRUCTURE COMMANDS
%	Skip this unless you know what you're doing
%----------------------------------------------------------------------------------------

% Header and footer for when a page split occurs within a problem environment
\newcommand{\enterProblemHeader}[1]{
%\nobreak\extramarks{#1}{#1 continued on next page\ldots}\nobreak
%\nobreak\extramarks{#1 (continued)}{#1 continued on next page\ldots}\nobreak
}

% Header and footer for when a page split occurs between problem environments
\newcommand{\exitProblemHeader}[1]{
%\nobreak\extramarks{#1 (continued)}{#1 continued on next page\ldots}\nobreak
%\nobreak\extramarks{#1}{}\nobreak
}

\setcounter{secnumdepth}{0} % Removes default section numbers
\newcounter{homeworkProblemCounter} % Creates a counter to keep track of the number of problems
\setcounter{homeworkProblemCounter}{0}

\newcommand{\homeworkProblemName}{}
\newenvironment{homeworkProblem}[1][Part \arabic{homeworkProblemCounter}]{ % Makes a new environment called homeworkProblem which takes 1 argument (custom name) but the default is "Problem #"
\stepcounter{homeworkProblemCounter} % Increase counter for number of problems
\renewcommand{\homeworkProblemName}{#1} % Assign \homeworkProblemName the name of the problem
\section{\homeworkProblemName} % Make a section in the document with the custom problem count
\enterProblemHeader{\homeworkProblemName} % Header and footer within the environment
}{
\exitProblemHeader{\homeworkProblemName} % Header and footer after the environment
}

\newcommand{\problemAnswer}[1]{ % Defines the problem answer command with the content as the only argument
\noindent\framebox[\columnwidth][c]{\begin{minipage}{0.98\columnwidth}#1\end{minipage}} % Makes the box around the problem answer and puts the content inside
}

\newcommand{\homeworkSectionName}{}
\newenvironment{homeworkSection}[1]{ % New environment for sections within homework problems, takes 1 argument - the name of the section
\renewcommand{\homeworkSectionName}{#1} % Assign \homeworkSectionName to the name of the section from the environment argument
\subsection{\homeworkSectionName} % Make a subsection with the custom name of the subsection
\enterProblemHeader{\homeworkProblemName\ [\homeworkSectionName]} % Header and footer within the environment
}{
\enterProblemHeader{\homeworkProblemName} % Header and footer after the environment
}

%----------------------------------------------------------------------------------------
%	NAME AND CLASS SECTION
%----------------------------------------------------------------------------------------

\newcommand{\hmwkTitle}{Assignment\ \#$\sqrt{-1}$} % Assignment title
\newcommand{\hmwkDueDate}{Friday,\ January\ 1,\ 2016} % Due date
\newcommand{\hmwkClass}{CSC320} % Course/class
\newcommand{\hmwkClassTime}{L0101} % Class/lecture time
\newcommand{\hmwkAuthorName}{Firstname Lastname} % Your name

%----------------------------------------------------------------------------------------
%	TITLE PAGE
%----------------------------------------------------------------------------------------

\title{
\vspace{2in}
\textmd{\textbf{\hmwkClass:\ \hmwkTitle}}\\
\normalsize\vspace{0.1in}\small{Due\ on\ \hmwkDueDate}\\
\vspace{0.1in}
\vspace{3in}
}

\author{\textbf{\hmwkAuthorName}}
%\date{} % Insert date here if you want it to appear below your name

%----------------------------------------------------------------------------------------

\begin{document}

\maketitle
\clearpage

%----------------------------------------------------------------------------------------
%	PART 1
%----------------------------------------------------------------------------------------
\begin{homeworkProblem}
\noindent \textit{Dataset description}
\begin{enumerate}
  \item variety of angles
  \item different styles of handwriting
  \item gaps between continuous lines
  \item different thickness levels
\end{enumerate}
\begin{center}
  \includegraphics[scale=0.3]{images/dataset.png}
\end{center}
\end{homeworkProblem}
%----------------------------------------------------------------------------------------
%	PART 2
%----------------------------------------------------------------------------------------
\begin{homeworkProblem}
\noindent \textit{Compute the network function by propagating forward and
  discarding intermediate results}
\begin{lstlisting}[language=python,numbers=left]
def compute_network(x, W0, b0, W1, b1):
    _,_, output = forward(x, W0, b0, W1, b1)
    return = argmax(output)
\end{lstlisting}
\end{homeworkProblem}
%----------------------------------------------------------------------------------------
%	PART 3 
%----------------------------------------------------------------------------------------
\begin{homeworkProblem}
  \begin{enumerate}
    \item We will use negative log-probabilities as our cost function, and find
      its gradient
      \begin{lstlisting}[language=python,numbers=left]
def cross_entropy(y, y_):
  return -sum(y_ * log(y))
      \end{lstlisting}
    \item Vectorized code for computing gradient of the cost function
      \begin{lstlisting}[language=python,numbers=left]
def deriv_multilayer(W0, b0, W1, b1, x, L0, L1, y, y_):
    dCdL1 = y - y_
    dCdW1 = dot(L0, dCdL1.T)
    dCdobydodh = dot(W1, dCdL1)
    one_minus_h_sq = 1 - L0**2

    dCdW0 = tile(dCdobydodh, 28 * 28).T * dot(x, (one_minus_h_sq.T))
    dCdb1 = dCdL1
    dCdb0 = dCdobydodh * one_minus_h_sq

    return dCdW1, dCdb1, dCdW0, dCdb0
      \end{lstlisting}
  \end{enumerate}
\end{homeworkProblem}
%----------------------------------------------------------------------------------------
%	PART 4 
%----------------------------------------------------------------------------------------
\begin{homeworkProblem}
      \begin{lstlisting}[language=python,numbers=left]
plot_iters = []
plot_performance = []
def train(plot=False):
    global W0, b0, W1, b1
    global plot_iters, plot_performance
    alpha = 1e-3
    for i in xrange(50):
        X, Y, examples_n = get_batch(i * 5)

        update = np.zeros(4)

        for j in xrange(examples_n):
            y = Y[j].reshape((10, 1))
            x = X[j].reshape((28 * 28, 1))
            L0, L1, output = forward(x, W0, b0, W1, b1)
            gradients = deriv_multilayer(W0, b0, W1, b1, x, L0, L1, output, y)
            update = [update[k] + gradients[k] for k in range(len(gradients))]

        # update the weights 
        W1 -= alpha * update[0]
        b1 -= alpha * update[1]
        W0 -= alpha * update[2]
        b0 -= alpha * update[3]
        if plot:
            plot_iters.append(i * examples_n)
            plot_performance.append(test_perf())
            
train(plot=True)
      \end{lstlisting}
      \begin{lstlisting}[language=python,numbers=left]
def get_batch(offset):
    # 5 examples per class
    example_per_class = 5
    classes_num = 10
    x_batch = np.zeros((example_per_class * classes_num, 28 * 28))
    y_batch = np.zeros((example_per_class * classes_num, classes_num))
    for i in xrange(classes_num):
        for j in xrange(example_per_class):
            x_batch[i * example_per_class + j] = M['train' + str(i)][j +
                                                                     offset]
            y_batch[i * example_per_class + j][i] = 1
    return x_batch, y_batch
      \end{lstlisting}
  \includegraphics[]{images/performance.png}
\end{homeworkProblem}

\end{document}